% ! TEX root = ../racket.tex
\chapter{Proust}
\label{ch:proust}
\index{Racket}
\index{Racket!modules}

This chapter will provide an example of a great use of Racket to craft
a \qq{nano proof assistant}, called Proust. The presentation closely
follows \cite{proust} and it will contain further clarifications as
needed. For a very basic Racket introduction, we have added in the
Appendix some special syntactic features and we also point the reader
to the excellent official documentation (\cite{racket}).

As the author mentions, the article is intended as a DIY approach
for a simple proof assistant (in fact, the name derives from a weird
contraction of the expression \qq{proof assistant}). The general goal
is much more interesting than that, in that it will also delve into the
inner works of the underlying mechanisms of proof asistants, for the
purpose of implementing them in a simpler manner.

Note that since Racket is rather a toolbox for crafting one's own toy
languages, any source code must start with a \texttt{\#\!\!\! lang} pragma,
which mentions what part of Racket one wants to use. Common pragmas
(formally, \emph{modules}) include:
\begin{itemize}
\item \texttt{htdp} -- the special module with syntax used in the
  \cite{htdp} book;
\item \texttt{racket} -- the full-fledged module with all syntax and features;
\item \texttt{racket/base} -- the simplest submodule of \texttt{racket};
\item \texttt{racket/typed} -- the module of Racket with strong typing;
\item \texttt{scheme} -- the backwards compatible module allowing one
  to use Scheme syntax.
\end{itemize}

To not overcomplicate the example and to skip any decision process,
we will just use \texttt{\#lang racket}. This will also mean that
we will be working in an \emph{untyped} environment, although we will
define custom syntax for explicit type annotations and functional types.

%%%%%%%%%%%%%%%%%%%%%%%%%%%%%%%%%%%%%%%%%%%%%%%%%%%%%%%%%%%%%%%%%%%%%%
\section{The Grammar and Basic Parsing}
\index{Racket!structures}
\index{Racket!structures!transparency}

First, we specify the language that we will use to make proof terms,
which will be a mix of untyped lambda calculus and simple and function
types. As such, the BNF grammar of the language is as follows:
\begin{align*}
  \texttt{expr} ::&= (\lambda x \Rightarrow \texttt{expr}) \\
                  &| \ (\texttt{expr} \ \texttt{expr}) \\
                  &| \ (\texttt{expr} : \texttt{type}) \\
                  &| \ x \\
  \texttt{type} ::&= (\texttt{type} \to \texttt{type}) \\
                  &| \ X.
\end{align*}

The reader will recognize, respectively: lambda abstraction, application,
type inhabitance (also known as typed variable declaration) and
free variables. These are the legal expressions and the types are
either simple or functional.

To recognize the structures that appear, we will use the standard
Racket structures (records), which will also allow pattern-matching to
extract the parts that are needed. This works more or less as a DIY
approach to typing, where we make things as verbose as possible, allowing
for easy recognition and pattern matching for such types:
{
  \small
\begin{verbatim}
(struct Lam (var body) #:transparent)         ; lambda abstractions
(struct App (func arg) #:transparent)         ; application
(struct Ann (expr type) #:transparent)        ; explicit type annotations
(struct Arrow (domain range) #:transparent)   ; functional types
\end{verbatim}
}

A little remark about \emph{transparent} vs.\ \emph{opaque} structures.
By default, all structures defined in Racket are opaque. What this means
for our purposes is that if one prints such a structure, it doesn't
show anything about its internal contents. For example, printing
a \texttt{Lam} will output \texttt{\#<Lam>} (see the code examples
that follow below). On the other hand, a transparent structure shows
its internals when printed, i.e.\ it will print something like
\texttt{(Lam 'x 'y)}.

Since by default, all Racket structures are opaque, transparency must
be enabled explicitly.

The first goal of the simple proof assistant will be to parse expressions,
in order to understand them appropriately and extract the needed parts.
For this, we define a function \texttt{parse-expr}, which takes a so-called
S-expression (standard Racket/Lisp expression) and produces an element
that is a legal \texttt{expr}.
{
  \small
\begin{verbatim}
(define (parse-expr s)
  (match s                              ; pattern-match s
    [`(lambda ,(? symbol? x) => ,e)     ; is it a lambda expression?
        (Lam x (parse-expr e))]         ; make it a Lam
    [`(,e0 ,e1)                         ; is it an application?
        (App (parse-expr e0)            ; make it an App
             (parse-expr e1))]
    [`(,e : ,t)                         ; is it a type annotation (e : t)?
        (Ann (parse-expr e)             ; make it an Ann
             (parse-type t))]
    [`(,e1 ,e2 ,e3 ,r ...)              ; is it a general list?
        (parse-expr 
            `((,e1 ,e2) ,e3 ,@r))]      ; parse it by parts
    [(? symbol? x) x]                   ; else, it's a symbol, return it
    [else (error 'parse "bad syntax: ~a" s)]))
\end{verbatim}
}
\index{Racket!quote}
\index{Racket!unquote}
\index{Racket!quasiquote}
\index{Racket!unqoute!splice}

Notice the very clever use of the quoting and unquoting (including splice)
mechanism (detailed in \ref{ch:a-racket}) in the definition of
\texttt{parse-expr}, as well as the square bracket delimitation of the matching
cases. The quote for the patterns ensures that
we are searching for expressions that are either (literally) \texttt{(lambda ...)}
or \texttt{(... ...)}, but inside the quoting we actually want to see
what's there, so we evaluate the components using the unquote.
The special predicate \texttt{(?\!\! symbol?\!\! x)} is used only in pattern
matching, where the first question mark matches anything (similar to
\texttt{*} in regular expressions). Also note that there is a convention
in Lisps to name predicates (i.e.\ functions that return true or false)
ending with a question mark. So basically the last successful case
of pattern matching is used for symbols (literals).

Most of the syntax should be clear and it is also accompanied by
comments which should ease understanding. The only new piece of syntax
which we comment on is the \emph{ellipse} (\texttt{...}), which is used
to match anything that follows (\qq{the rest}).
\index{Racket!ellipse}

Similarly, we parse type expressions:
{
  \small
\begin{verbatim}
(define (parse-type t)
  (match t
    [`(,t1 -> ,t2)                              ; is it a functional type?
        (Arrow (parse-type t1)                  ; make it an Arrow
                (parse-type t2))]
    [`(,t1 -> ,t2 -> ,r ...)                    ; maybe it's a multi-arg function
        (Arrow (parse-type t1)
               (parse-type `(,t1 -> ,@r)))]
    [(? symbol? X) X]                           ; is it a simple type? return it
    [else (error "can't parse this type")]))
\end{verbatim}
}

To make printing clear enough, we define helper functions that do
\qq{unparsing}, both for expressions and for types:
{
  \small
\begin{verbatim}
;; pretty-print-expr : expr -> String
(define (pretty-print-expr e)
  (match e
    [(Lam x b) (format "(lambda ~a => ~a)" x (pretty-print-expr b))]
    [(App e1 e2) (format "(~a ~a)"
                          (pretty-print-expr e1)
                          (pretty-print-expr e2))]
    [(? symbol? x) (format "~a" x)]
    [(Ann e t) (format "(~a : ~a)"
                        (pretty-print-expr e)
                        (pretty-print-expr t))]))

;; pretty-print-type : Type -> String
(define (pretty-print-type t)
  (match t
    [(Arrow t1 t2) (format "(~a -> ~a)"
                            (pretty-print-type t1)
                            (pretty-print-type t2))]
    [else t]))
\end{verbatim}
}

Notice another piece of new syntax in formatted printing,
where the \texttt{~a} placeholder is used, i.e.\ it will be
replaced by the arguments that follow. For example, the syntax
\texttt{(format "\~\!a" 'x)} will put the symbol \texttt{x}
in place of \texttt{"\~\!a"} when printing.

When evaluating and checking proofs, we will need a \emph{context},
which is defined as a \texttt{(listof (List Symbol Type))}. The
\texttt{listof} keyword is called a \emph{contract} in this case and
in short, it \emph{asserts} that what it expects is a list with a
symbol and a type. If the context does not respect the contract,
we get a specific error. Racket provides extensive support for software
contracts, for various items such as structures, functions, variables,
but we do not actually need any such complexity here, so we will
only indicate \cite{contract} for further information. Also, for
(judgments in) contexts, we refer the reader to our \S\ref{sec:depty},
where we see that a context $ \Gamma $ is basically made of lists of
typed variable declarations of the form $ x : A $.
\index{Racket!contract}

Now, given the fact that a context is a list of a symbol and a type,
we can \texttt{pretty-print} it as well using the function:
{
  \small
\begin{verbatim}
(define (pretty-print-context ctx)
  (cond
    [(empty? ctx) ""]
    [else (string-append (format "\n~a : a"
                                 (first (first ctx))
                                 (pretty-print-type (second (first ctx))))
          (pretty-print-context (rest ctx)))]))
\end{verbatim}
}

%%%%%%%%%%%%%%%%%%%%%%%%%%%%%%%%%%%%%%%%%%%%%%%%%%%%%%%%%%%%%%%%%%%%%%
\section{Checking Lambdas}

Now, given this setup, we can actually start writing the functions
for the proofs. \textbf{We remark explicitly that for the purposes %
  of this chapter, we will only present the part that uses lambda terms
  for proofs.}

First, type-checking, which checks whether an
expression has a certain type in a given context.
{
  \small
\begin{verbatim}
;; type-check : Context Expr Type -> Boolean
;; produces true if expr has type t in context ctx
;; (else, error)
(define (type-check ctx expr type)
  (match expr
    [(Lam x t)              ; is expr a lambda expression?
      (match type           ; then the type must be an arrow
        [(Arrow tt tw) (type-check (cons `(,x ,tt) ctx) t tw)]
        [else (cannot-check ctx expr type)])]
  [else (if (equal? (type-infer ctx expr) type) true    ; fail for other types
            (cannot-check ctx expr type))]))

;; the error function
(define (cannot-check ctx expr type)
  (error 'type-check "cannot typecheck ~a as ~a in context:\n~a"
    (pretty-print-expr expr)
    (pretty-print-type type)
    (pretty-print-context ctx)))
\end{verbatim}
}

Then the function that tries to make a type inference.
{
  \small
\begin{verbatim}
;; type-infer : Context Expr -> Type
;; tries to produce type of expr in context ctx
;; (else, error)
(define (type-infer ctx expr)
  (match expr
    [(Lam _ _) (cannot-infer ctx expr)]     ; lambdas are handled in type-check
    [(Ann e t) (type-check ctx e t) t]      ; check type annotations
    [(App f a)                              ; function application
        (define tf (type-infer ctx f))
          (match tf                         ; must be arrow type
            [(Arrow tt tw) #:when (type-check ctx a tt) tw] ; when the rest typechecks
            [else (cannot-infer ctx expr)])]
    [(? symbol? x)                          ; for symbols
       (cond
         [(assoc x ctx) => second]          ; if it's a list, the second is the type
         [else (cannot-infer ctx expr)])])) ; else, not okay

;; the error function
(define (cannot-infer ctx expr)
  (error 'type-infer "cannot infer type of ~a in context:\n~a"
    (pretty-print-expr expr)
    (pretty-print-context ctx)))
\end{verbatim}
}

%%%%%%%%%%%%%%%%%%%%%%%%%%%%%%%%%%%%%%%%%%%%%%%%%%%%%%%%%%%%%%%%%%%%%%
\section{Basic Testing}
\index{Racket!testing}

Now we can define a function that checks some basic proofs like so:
{
  \small
\begin{verbatim}
(define (check-proof p)
    (type-infer empty (parse-expr p)) true)
\end{verbatim}
}

This way, if errors do not appear, the function will successfully apply
the \texttt{type-infer} procedure and return \texttt{true}, which can be
seen as a \qq{successful exit code}.

Using this, we can use the Racket testing module to check some basic
proofs, such as:
{
  \small
\begin{verbatim}
(require test-engine/racket-tests)              ; import the test module

;; check whether check-proof returns true => good proof
;; lambda xy.x : A -> (B -> A)
(check-expect
  (check-proof '((lambda x => (lambda y => x)) : (A -> (B -> A))))
    true)

;; lambda xy.yx : (A -> ((A -> B) -> B))
(check-expect
  (check-proof '((lambda x => (lambda y => (y x))) : (A -> ((A -> B) -> B))))
    true)

;; lambda fgx.f(gx) : ((B -> C) -> ((A -> B) -> (A -> C)))
(check-expect
  (check-proof '((lambda f => (lambda g => (lambda x => f (g x)))) :
                    ((B -> C -> ((A -> B) -> (A -> C))))))
    true)

(test)          ;; => All 3 tests passed.
\end{verbatim}
}

\vspace{0.3cm}

Formally, what \texttt{check-proof} does is to verify well-formedness of
typing judgments for lambda expressions (see our \S\ref{ch:mltt}). As such,
for example, the first proof checks the typing for the Church Boolean
\texttt{true} in our \S\ref{sec:unty-lambda}.

The other two proofs are just simple exercises. We can also write, for example,
the typing for the Church numeral $ c_1 $ (again, see \S\ref{sec:unty-lambda}).
Given that:
\[
  c_1 = \lambda s . \lambda z . sz = \lambda sz.sz,
\]
we have:
{
  \small
\begin{verbatim}
;; c1 = lambda sz . sz : N -> N
(check-expect
  (check-proof '((lambda s => (lambda z => (s z))) : (N -> N))))
(test)                  ;; => All 1 test passed.
\end{verbatim}
}


%%% Local Variables:
%%% mode: latex
%%% TeX-master: "../racket"
%%% End:
