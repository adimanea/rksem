% ! TEX root = ../racket.tex
\chapter{Proust}

This chapter will provide an example of a great use of Racket to craft
a \qq{nano proof assistant}, called Proust. The presentation closely
follows \cite{proust} and it will contain further clarifications as
needed.

As the author mentions, the article is intended as a DIY approach
for a simple proof assistant (in fact, the name derives from a weird
contraction of the expression \qq{proof assistant}). In fact, the goal
is much more interesting than that, in that it will also delve into the
inner works of the underlying mechanisms of proof asistants, for the
purpose of implementing them in a simpler manner.

Note that since Racket is rather a toolbox for crafting one's own toy
languages, any source code must start with a \texttt{\# lang} pragma,
which mentions what part of Racket one wants to use. Common pragmas
(formally, \emph{modules}) include:
\begin{itemize}
\item \texttt{htdp} -- the special module with syntax used in the
  \cite{htdp} book;
\item \texttt{racket} -- the full-fledged module with all syntax;
\item \texttt{racket/base} -- the simplest submodule of \texttt{racket};
\item \texttt{racket/typed} -- the module of Racket with strong typing;
\item \texttt{scheme} -- the backwards compatible module allowing one
  to use Scheme syntax.
\end{itemize}

To not overcomplicate the example and to skip any decision process,
we will just use \texttt{\#lang racket}.

First, we specify the language that we will use to make proof terms,
which will be a mix of untyped lambda calculus and simple and function
types. As such, the BNF grammar of the language is as follows:
\begin{align*}
  \texttt{expr} ::&= (\lambda x \Rightarrow \texttt{expr}) \\
                  &| \ (\texttt{expr} \ \texttt{expr}) \\
                  &| \ (\texttt{expr} : \texttt{type}) \\
                  &| \ x \\
  \texttt{type} ::&= (\texttt{type} \to \texttt{type}) \\
                  &| \ X.
\end{align*}

The reader will recognize, respectively: lambda abstraction, application,
type inhabitance and free variables. These are the legal expressions
and the types are either simple or functional.

To recognize the structures that appear, we will use the standard
Racket structures (records), which will also allow pattern-matching to
extract the parts that are needed:
{
  \small
\begin{verbatim}
(struct Lam (var body))         ; lambda abstractions
(struct App (func arg))         ; application
(struct Arrow (domain range))   ; functionals
\end{verbatim}
}

The first goal of the simple proof assistant will be to parse expressions,
in order to understand them appropriately and extract the needed parts.
For this, we define a function \texttt{parse-expr}, which takes a so-called
S-expression (standard Racket/Lisp expression) and produces an element
that is a legal \texttt{expr}.
{
  \small
\begin{verbatim}
(define (parse-expr s)
  (match s                              ; pattern-match s
    [`(lambda ,(? symbol? x) => ,e)     ; is it a lambda expression?
        (Lam x (parse-expr e))]         ; make it a Lam
    [`(,e0 ,e1)                         ; is it an application?
        (App (parse-expr e0)            ; make it an App
             (parse-expr e1))]
    [(? symbol? x) x]))                 ; else, it's a symbol, return it
\end{verbatim}
}

Notice the very clever use of the quoting and unquoting mechanism in the
definition of \texttt{parse-expr}, as well as the square bracket delimitation
of the matching cases. The quote for the patterns ensures that we are searching
for expressions that are either (literally) \texttt{(lambda ...)} or \texttt{(... ...)},
but inside the quoting we actually want to see what's there, so we
evaluate the components using the unquote. The special predicate
\texttt{(? symbol? x)} is used only in pattern matching, where the first
question mark matches anything (similar to \texttt{*} in regular expressions).
Also note that there is a convention in Lisps to name predicates (i.e.\ functions
that return true or false) ending with a question mark.

Similarly, we parse type expressions:
{
  \small
\begin{verbatim}
(define (parse-type t)
  (match t
    [`(,t1 -> ,t2)                              ; is it a functional type?
        (Arrow (parse-type t1)                  ; make it an Arrow
                (parse-type t2))]
    [(? symbol? X) X]                           ; is it a simple type? return it
    [else (error "can't parse this type")]))
\end{verbatim}
}


%%% Local Variables:
%%% mode: latex
%%% TeX-master: "../racket"
%%% End:
