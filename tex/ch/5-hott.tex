% ! TEX root = ../racket.tex
\chapter{Some HoTT Pie by the Book}

...or rather \emph{some Pie by the HoTT book!} In this final chapter, we make
a basic attempt at implementing some of the homotopy type theory features in
Pie. The theory will closely follow \cite{hott} (also known as
\qq{The HoTT book}) and some of the references therein, while the implementation
part will be made along the lines of \cite{typer}, along with the standard
Racket documentation \cite{racket}.

For the purposes of this dissertation, we will limit ourselves to the notions
introduced in the first chapter of the HoTT book, with the hope that more
work will follow. The presentation will be structured by starting with the
main theoretical aspects of each item that will be presented, from a mathematical
and type-theoretical point of view and then the Pie implementation. If the
item is already part of the Pie standard library, we will present and explain
it, otherwise, we will detail our take on it.

Finally, we note that to the best of our knowledge at the moment of writing
this thesis, there are no similar attempts of homotopy type theory in Racket
or any domain-specific language built with it. The standard, well-known
approaches feature either Coq, via \emph{The Coq HoTT Project} (\cite{coqhott})
or Agda (\cite{agdahott}).

%%%%%%%%%%%%%%%%%%%%%%%%%%%%%%%%%%%%%%%%%%%%%%%%%%%%%%%%%%%%%%%%%%%%%%

\section{Preamble: Set Theory and Type Theory}

Before we get to the actual implementation, we make a short stop to discuss
some specific differences between set theory and type theory. We should point
out that many of the type-theoretical and logical aspects of homotopy type
theory (hereafter referred to as HoTT) are taken from Per Martin-L\"{of}'s
work, some of which we touched in the first chapter of the dissertation.
Nevertheless, given the complexity and the size of the HoTT project, multiple
changes have appeared, some more radical, some only to modernize the
presentation. Hence, although this discussion could well belong to
Martin-L\"{o}f's theory, we chose to present it here, as it is more coherent
with the (modern) HoTT.

The main starting point is that type theory aims to provide foundational
results for mathematics, overcoming the difficulties that set theory was shown
to contain\footnote{On a similar note, \emph{topos theory} is another approach
  that has a similar goal, coming from topology and category theory. While the
  subject is way beyond the scope of this dissertation, we mention it and refer
  the interested reader to the excellent book \cite{macsh} or the more introductive
  \cite{goldb}. A historical overview of the topic is presented, for example,
  in \cite{abuse}.}. For this reason, many discussions have been sparked as to
what differentiates type theory from all the previous attempts of formalizing the
foundations of mathematics in set theory. Without going into technical,
complex, discussions, we mention some basic points that are relevant to the
current presentation. We rely for this on the overview presented in
\cite[\S 1.1]{hott}.

First, it should be pointed out that set theory, as used in the foundational
research, is comprised of two parts:
\begin{itemize}
\item the deductive layer of first order logic, that is used to reason, i.e.\
  to write proofs;
\item the axioms that are formulated in this system and which regulate
  the objects that live in the theory. In this case, the Zermelo-Fraenkel
  with the Choice Axiom (usually written as ZFC) is the most used
  axiomatic system.
\end{itemize}
It follows that set theory is not actually about sets \emph{per se}, but
rather about the interactions between the sets, as objects instantiated
by the second (axiomatic) layer by the rules of the first (deductive) layer,
which help us make propositions.

On the other hand, type theory is built upon itself, in a way. What we
mean is that type theory is its own deductive system. Whereas set theory
is based on \emph{sets} and \emph{propositions} (about sets), type theory
is only about \emph{types}. As such, the primitive notion of type can be
subject to various interpretations which become (parts of) mathematical
theories: some types can be seen as propositions, some as sets, some as
topological spaces, categories and so forth. It follows that the mathematical
(\emph{formal}) act of \emph{proving a theorem} inside set theory becomes the
\emph{constructive} (i.e.\ intuitionistic) act of \emph{constructing an object},
namely an inhabitant of a type.

Another key difference is related to the structure of deductions themselves.
In general, we can make a deductive system from (inference) \emph{rules}
and \emph{judgments}, which are derived by the rules. In the case of
first order logic, which regulates set theory, the basic judgment is
that a proposition has a proof. From it, one can construct derived proofs,
such as, for example, the proof of $ A \land B $, which is made of a proof
of $ A $ in conjunction with a proof of $ B $, by a rule of proof construction.

On the other hand, the basic judgment of type theory is of the form $ a : A $,
read \qq{\emph{element $ a $ has type $ A $}}. But this judgment is analogous
to the provability of proposition $ A $, were it to be interpreted as a type.
That is, exposing an inhabitant $ a $ of the type $ A $ \emph{is not} similar to
showing an element of a set $ a \in A $. Membership is taken for granted in type
theory when we make a typing judgment. That is, the judgment $ a : A $ cannot
be proved or disproved so it cannot be used in composite judgments such as
\emph{if $ a : A $ then $ b : B $}. Intuitively, this is because whenever we
want to model the use of a certain element, we think of its \qq{natural type},
i.e.\ one which we are sure it inhabits, then start from there. We never,
even in common mathematical speech, start with something like
\emph{Let $ a = \dfrac{\sqrt[3]{1 + \sqrt[5]{5}}}{\ln 7} : \NN $} if we are not sure
that it is true first. Usually, if we need to use $ a $ in some proof,
we start with it as an inhabitant of a type which we are sure it belongs to,
e.g.\ \emph{let $ a : \RR $}\footnote{Note that we have not provided examples
  of types yet and if this example induces the confusion that all sets are
  types or vice versa (or at least, numeric sets), feel free to skip this
  discussion, as we will not insist on it anyway.}.

Lastly, a very important difference between sets and types is the treatment
of \emph{equality}. In common set theory, equality is a proposition, meaning
that it can be proved or disproved for various sets or elements. In type theory,
however, \emph{equality is a type}. That is, taking, say $ a, b : A $, there
exists the type $ a =_A b $, which is inhabited if and only if $ a $ and $ b $
represent the same element. Before going further, a quick detour: What would
an inhabitant of an equality type look like? What \qq{instantiates} the fact
that is commonly written as $ a = b $? This is not an easy question and we only
mention that this is one of the starting points of the \emph{homotopy} part of
HoTT. That is, using Martin-L\"{o}f's brilliant idea of making equality a type,
homotopies (formally known as \emph{homotopy maps}) entered the scene. They
provide a kind of \qq{relaxed} form of equality by means of, e.g.\ in topology,
\emph{continuous deformations}. Since in topology, continuity is one of the
most important properties, if a space can be continuously transformed into
another one, we simply say that \qq{they are the same}\footnote{This is the
  origin of the famously funny \qq{facts} which state that
  \href{https://upload.wikimedia.org/wikipedia/commons/2/26/Mug\_and\_Torus\_morph.gif}{%
    a bagel is the same as a cup} or
  \href{https://upload.wikimedia.org/wikipedia/commons/2/24/Spot\_the\_cow.gif}{%
    a cow is the same as a sphere}.}
What this example shows is that there are at least some interpretations of
type theory where (the local notion of) equality can have witnesses, which
are, in this case, \emph{homotopy maps}.
\index{types!equality}
\index{types!equality!propositional}
\index{types!equality!definitional}
\index{types!equality!judgmental}
\index{types!homotopy}

Back to the original discussion, when the type $ a =_A b $ is inhabited,
we say that the elements $ a $ and $ b $ are \emph{propositionally equal}.
That is, the fact that an equality type is inhabited provides an example
of propositional equality. But there is another kind of equality, called
\emph{judgmental} or \emph{definitional} equality, which is more akin to the
common mathematical sense. For example, $ 3^2 $ and $ 9 $ are equal by definition
(of multiplication, in this case), so they exhibit a definitional equality.
It is \emph{judgmental} in the sense that it can be proved, e.g.\ by using
the definition of multiplication in the example we provided. We finish by
mentioning that this kind of equality can be used for whole function equality,
for example, by means of the distinction between intensional and extensional
concepts which we mentioned in Remark \ref{rk:int-ext}. For example,
the function $ x \mapsto (x + 2) $ is \emph{judgmentally equal} to
the function $ y \mapsto (y + 3 - 1) $. Usually, definitional equality is
denoted by $ \equiv $, so $ 3^2 \equiv 9 $ etc.

%%%%%%%%%%%%%%%%%%%%%%%%%%%%%%%%%%%%%%%%%%%%%%%%%%%%%%%%%%%%%%%%%%%%%%

% \section{Function Types}

% Although we have touched on the subject of functions in type theory in
% \S\ref{sec:unty-lambda} and \S\ref{sec:dep-fun} (for the dependent case),
% we revisit the subject for a treatment in the formulation of HoTT.

% We start by noting that in HoTT (as in the case of multiple other type
% theories), \emph{functions are primitive}. That is, whereas in set theory
% they are defined as subsets of a Cartesian product, in type theory they
% are not defined, just introduced and used.

% We could prescribe the rules that are specific to function types in the
% form of proof trees as in \S\ref{ch:mltt}, but we prefer a more explanatory
% approach.

% A first use of functions is through \emph{application}. Formally, this means
% that we have an \emph{elimination rule}: given a function type $ A \to B $,
% inhabited by some $ f : A \to B $ and some inhabitant $ a : A $, we
% obtain an inhabitant $ f(a) : B $, which we call the \emph{application (evaluation)}
% of $ f $ in $ a $. Note also that it is not uncommon to write this as $ fa $.

% An \emph{introduction rule} for function types shows us how to construct
% inhabitants of the type, which in consequence, will instantiate the type.
% This can be done either explicitly, i.e.\ by mentioning an equation that
% contains the inhabitant of the domain or not (for example, $ f : A \to B, %
% f(x) = 2x $ and we call $ E(x) $ the equation that defines the function)
% or via \emph{typed} $ \lambda $-abstraction, e.g.\
% $ \left(\lambda(x : A) . 2x\right) : A \to B $.

% A definitional equality shows how to use lambda abstraction, thus making
% a \emph{computation rule}:
% \[
%   (\lambda x . E(x))(a) \equiv E(a/x),
% \]
% where $ E(a/x) $ means the expression $ E $ where every occurrence of $ x $
% has been replaced by $ a $. This rule is known in lambda calculus by the
% name of \emph{$\beta$-reduction}, which we have met before in \S\ref{sec:dep-pi}.

% Furthermore, what in $ \lambda $-calculus is known as $ \eta $-conversion
% (see also \S\ref{sec:dep-pi}) we call \emph{the uniqueness principle for %
%   function types}. This states that a function is completely defined by its
% values, thus making an extensional definition.

% Lastly, we also have a judgmental equality that basically spells out
% the rule known as $ \alpha $-conversion. This states that the bound variable
% in a lambda expression is a dummy variable, which means that, for example,
% $ \lambda (x : A).E(x) \equiv \lambda (y : A).E(y/x) $.

% \vspace{0.3cm}

% As for the Pie implementation of function types, they are built in
% (in fact, inherited directly from Typed Racket, namely
% the \texttt{\#lang typed/racket} module). As such, a function type is written as
% \texttt{(-> A B)} and its inhabitants are \texttt{claim}ed accordingly,
% while the definition will make use of a \texttt{lambda} expression.

% For example:
% {
%   \small
% \begin{verbatim}
% #lang pie

% (claim twice (-> Nat Nat))
% (define twice
%   (lambda (x) (+ x x)))             ; assume Nat addition has been defined
% \end{verbatim}
% }

% Given the \texttt{claim}, the definition automatically coerces \texttt{x} to
% be of type \texttt{Nat}, but we could have easily written it in Typed Racket,
% which doesn't need the \texttt{claim}, but makes the type explicit in the
% definition:
% {
%   \small
% \begin{verbatim}
% #lang typed/racket

% (define twice
%   (lambda ([x : Natural]) (+ x x)))     ; + is built-in
% \end{verbatim}
% }

% We close this by remarking that the Pie snippet is, in some sense, clearer by
% virtue of the \texttt{claim}, but keeping in mind that \texttt{lambda} is actually
% the \emph{type constructor} for functional types, both implementations should
% be equally easy to understand.

% \vspace{0.3cm}

% Finally, we also have \emph{dependent function types}, which we will completely
% skip here, as they have been the subject of \S\ref{sec:dep-pi}.

%%%%%%%%%%%%%%%%%%%%%%%%%%%%%%%%%%%%%%%%%%%%%%%%%%%%%%%%%%%%%%%%%%%%%%
\section{Product Types}

Before getting into this subject, note that we skipped the presentation of
function types and dependent ($\Pi$) function types, which have been the subject
of \S\ref{sec:dep-fun} and \S\ref{sec:dep-pi}, respectively. As such, we will
focus on new types, assuming the others presented, constructed and understood.

Product types follow the categorical notion of product in an arbitrary
category and generalize the Cartesian product for sets. As such, the basic
construction is denoted by $ A \times B $, for objects $ A, B $.

In type theory, we start with types $ A, B : \kal{U} $ (both having the
universe type, $ \kal{U} $) and we \emph{introduce} the type
$ A \times B : \kal{U} $, called their \emph{(Cartesian) product type}.
In particular, the nullary product leads to the \emph{unit type},
$ \pmb{1} : \kal{U} $. Canonical elements of type $ A \times B $
are \emph{pairs} $ (a, b) : A \times B $, where $ a : A $ and $ b : B $.
The only element of type $ \pmb{1} $ is denoted $ \ast : \pmb{1} $.

Note that since functions are primitive for type theory, ordered pairs
are also primitive. So we assume that for arbitrary types $ A, B $,
which are inhabited by some $ a : A, b : B $ say, we know what
$ (a, b) : A \times B $ means. This serves as the \emph{introduction rule}
for the product type, with the added remark that there is only one way
of constructing the sole inhabitant  $ \ast :  \pmb{1} $.

Now we want to see \emph{eliminators} for this type, i.e.\ how it is used.
Hence, we need to construct some non-dependent function $ f : A \times B \to C $
to inhabit the function type $ A \times B \to C $ and whose equations or
lambda-abstractions will show us what to do with pairs. But we know how to
form function types, so we form the type $ A \to B \to C $, parsed as
$ A \to (B \to C) $ and it follows that one inhabitant $ g $of this type is a
function that associates to each inhabitant $ a : A $ an inhabitant of
$ B \to C $, say $ g(a) $ which in turn associates to each inhabitant of $ B $ an
inhabitant of $ C $, i.e.\ $ g(a)(b) : C $.

Following this, we use the definitional equality:
\[
  f: A \times B \to C, \quad f(a, b) :\equiv g(a)(b),
\]
which is the elimination rule for products.

Two particular cases of interest are the \emph{canonical projections}:
\begin{align*}
  p_1 & : A \times B \to A, \quad p_1(a, b) :\equiv a \\
  p_2 & : A \times B \to B, \quad p_2(a, b) :\equiv b.
\end{align*}

\vspace{0.3cm}

As for the Pie implementation, this is exactly the \texttt{Pair} type, which
is built-in the language. Moreover, given the associativity of iterated
Cartesian product, i.e.\ $ A \times B \times C = (A \times B) \times C $,
generalized inductively, we get arbitrary products as \texttt{Pair}s of
\texttt{Pair}s. Another way of generalizing this is via the \texttt{List}
type. Hence, we refer the reader to our \S\ref{sec:pair-ex} and \S\ref{sec:inductive-ex}.

\todo[inline,noline,backgroundcolor=green!40]{add \texttt{rec}, \texttt{uniq} and \texttt{ind}?}

%%%%%%%%%%%%%%%%%%%%%%%%%%%%%%%%%%%%%%%%%%%%%%%%%%%%%%%%%%%%%%%%%%%%%%
\section{Dependent Pair Types ($ \Sigma $)}


%%% Local Variables:
%%% mode: latex
%%% TeX-master: "../racket"
%%% End:
