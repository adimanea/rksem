% ! TEX root = ../racket.tex

\chapter*{Appendix: Basic Racket Syntax}
\label{ch:a-racket}

Historically speaking, Racket is based on Scheme, being formerly called
PLT Scheme. As such, it expands on the standards R5RS (1998) and R6RS (2009)
of Scheme and diverge more significantly from R7RS (2013). However, we will
not be concerned about the differences between the languages and we will
assume that all Racket syntax that we introduce is valid Scheme syntax and
vice versa, unless explicitly mentioned otherwise.

Both languages emerged from a common ancestor, Lisp and as such, they
use a \emph{list syntax}. Furthermore, they use a prefix notation
for the functions, predicates and mathematical operations, the so-called
Polish ({\L}ukasiewicz) notation.
\index{Racket!Polish syntax}

The only delimiters Scheme uses are parentheses (and double quotations
for strings) whereas Racket strongly suggests the use of square brackets
inside parentheses for delimiting more important statements\footnote{Some %
  Scheme implementations allow the use of square brackets as well, but
  do not enforce it and furthermore, there are interpreters that reject
  this syntax.}

Loosely speaking, both Scheme and Racket are untyped, but they do recognize
three basic types: \emph{lists, functions} and \emph{symbols}, the
latter encompassing variables and \qq{everything else}.

A specific feature is the so-called \emph{quoting mechanism}, denoted
by a single quote (\texttt{'}) or a backtick (\texttt{`}) which
turns anything into a symbol and the reverse, the \emph{unquoting mechanism},
denoted by a comma (\texttt{,}) which enforces evaluation, as in the
case of a call by name versus call by value approach. For example,
\texttt{'(+ 1 2)} is interpreted as the symbol (string) \texttt{(+ 1 2)},
whereas \texttt{,(+ 1 2)} means \texttt{3}.
\index{Racket!quote}
\index{Racket!unquote}
\index{Racket!quasiquote}
\index{Racket!unquote!splice}

For simple expressions, quoting can be done either with a single
quote or with a backtick. However, in more complex expressions,
a difference appears. The backtick is actually called
\emph{quasiquote} and it can be used in expressions containing
unquote. So basically we have:
{
  \small
\begin{verbatim}
'(+ 1 2)                ;; => (+ 1 2)

(define x 3)
`(+ 2 ,x)               ;; => 5     (used quasiquote)
'(+ 2 ,x)               ;; error    (used quote)
\end{verbatim}
}

Moreover, there is also the \emph{splice unquote} syntax, which
is used to unquote lists. Instead of returning the whole list,
as a regular unquote would do, splice unquote, denoted by
\texttt{,{@}}, actually inserts the elements of the list:
{
  \small
\begin{verbatim}
(define x '(1 2 3))
`(+ 4 ,@x)              ;; => 10
\end{verbatim}
}
Notice also the use of the quote in the first line, because we
are just defining a (literal) list to store in \texttt{x}, whereas
in the second line, we used the quasiquote, since the expression
contains an unquote (the spliced one).

Another aspect of syntax is that a semicolon is used for comments and
the convention is to use a single semicolon for an inline comment
and two or more semicolons for comments spanning multiple lines.
The output of a program is commonly written as \texttt{;; => output}
inside the source code or documentation.

\begin{remark}\label{rk:implementation}
  A short word on implementation: all the examples that we will be
  showcasing, as well as the included source code is tested on a
  Manjaro Linux operating system using the Emacs 26 editor, the
  included Scheme mode and the third party
  \href{https://www.racket-mode.com/}{Racket mode}.

  After writing the source code, \texttt{C-c} loads the file into
  the respective REPL.
  \index{Racket!Emacs mode}
\end{remark}

Some simple examples follow.
{
  \small
\begin{verbatim}
(+ (* 5 3) 1)       ;; => 16

(define (mod2 x)
  (lambda (x) (rem x 2)))
(mod2 5)            ;; => 1

(define (mod3 x)
  "Write the remainder of x when divided by 3"
  (cond                                         ; multiple branching
     ((equal (rem x 3) 1) write "it's 1")       ; if (x % 3 == 1)
     ((equal (rem x 3) 2) write "it's 2")       ; if (x % 3 == 2)
     (#t write "it's 0")))                      ; default (true) case

(define (add-or-quote x)
  "Add 3 if x is even or write 'hello' else"
  (cond
    ((equal (mod2 x) 0) (lambda (x) ,(+ 1 2)))
    (#t (lambda (x) 'hello))))

(set sum '(+ 1 2 3))        ; defines the variable "sum"
(set x (* ,sum 2))          ; defines x to be 12
\end{verbatim}
}

\vspace{0.3cm}
\index{Racket!\texttt{cond}}

Hopefully, the rest of the syntax can be understood directly from
the examples that will follow. For further investigation, we recommend
the official Racket documentation \cite{racket} and the book
\cite{htdp}. As the need requires, we will also further explain the syntax.

%%% Local Variables:
%%% mode: latex
%%% TeX-master: "../racket"
%%% End:


%%% Local Variables:
%%% mode: latex
%%% TeX-master: "../racket"
%%% End:
