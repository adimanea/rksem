% ! TEX root = ../redex.tex
\chapter{My Examples}

{\color{red} \indent\indent \textbf{IDEA 1:}
  \begin{itemize}
  \item See Typed Racket for how it handles types;
  \item Take some (reasonably sophisticated) examples of type theory from \emph{Pierce, TAPL}
    (e.g.\ recursive types, subtyping);
  \item Implement them in Redex and verify stuff.
  \end{itemize}

  ---------------------------------

  \textbf{IDEA 2:}
  \begin{itemize}
  \item See PML - \emph{Intuitionistic Type Theory};
  \item See its pragmatic part in \emph{Programming in MLTT};
  \item Write as much as possible of it in Redex/Racket and verify stuff.
  \end{itemize}

  ---------------------------------

  \textbf{IDEA 3:}
  \begin{itemize}
  \item See ACL2 (e.g.\ \emph{Kaufmann - Computer Aided Reasoning});
  \item Write parts of it in Redex and verify stuff.
  \end{itemize}

  ---------------------------------

  \textbf{IDEA 4:}
  \begin{itemize}
  \item Take a (reasonably simple) algebraic theory, such as category theory,
    group theory or even sheaf theory;
  \item Look for an Agda/Coq implementation;
  \item Translate it in Redex/Racket and verify stuff;
  \item The plus side: if it doesn't go well, I can comment why Redex is (not)
    suited as a proof assistant (also goes for the ACL2 idea).
  \end{itemize}

  ---------------------------------

  \textbf{IDEA 5:}
  \begin{itemize}
  \item Take some chapters from the Program Verification course;
  \item Implement them in Redex;
  \item Verify examples;
  \item If it doesn't go well, I can comment why Redex is (not) suited as
    a program verification tool.
  \end{itemize}

  \todo[inline,noline,backgroundcolor=green!40]{Am I missing essential stuff? Is
  the \emph{reduction relations} part essential? My ideas focus most on judgments.}
  
}

%%% Local Variables:
%%% mode: latex
%%% TeX-master: "../redex"
%%% End:
