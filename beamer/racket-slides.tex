\documentclass[xcolor=dvipsnames,handout]{beamer}

% font setup
\usepackage{libertine}
\renewcommand*\familydefault{\sfdefault}    % Linux Libertine = default sans serif
\usepackage{inconsolata}                    % Inconsolata = monospaced
\usepackage[utf8]{inputenc}
\usepackage[T1]{fontenc}
\usepackage{tikz}
\usetikzlibrary{positioning}
\usepackage[style=german]{csquotes}
\newcommand\qq{\enquote}
\newcommand\dr{\mathrm}
\usepackage[normalem]{ulem}

\usepackage{ebproof}

\usepackage{algorithm}
\usepackage{algpseudocode}
\makeatletter
\renewcommand{\ALG@name}{Algoritm}
\makeatother
\algrenewcommand\algorithmicprocedure{\textbf{procedură}}
\algrenewcommand\algorithmicend{\textbf{final}}

% Graphics and other packages
\usepackage[romanian]{babel}
\usepackage{graphicx}
\addto\captionsromanian{\renewcommand{\figurename}{Ilustrație}}
\usepackage{caption}
\usepackage{subcaption}
\usepackage[style=german]{csquotes}

% Custom macros
\newcommand{\bloc}[3]{\begin{bl}<#1->{{\large\color{Gray}{\hrulefill}}\\ \color{bleumarin}{\large \emph{#2}}}\\ \vspace*{-2mm}{\color{Gray}{\hrulefill}}\\ #3 \end{bl}} 
\newcommand{\fr}[1]{\frame{#1}}
\newcommand{\ft}[1]{\frametitle{\color{bleumarin}{\hfill #1 \hfill}}}
\newcommand{\lin}[3]{\uncover<#1->{\alert<#1>{#2}}{\vspace*{#3 ex}}}
\newcommand{\ite}[2]{\uncover<#1->{\alert<#1>{\item #2}}}
\newcommand{\vs}[1]{\vspace*{#1 ex}}
\definecolor{bleumarin}{RGB}{30,30,150} 
\definecolor{firebrick}{RGB}{178,34,34}

% Theme setup
\useoutertheme{shadow} 
\usetheme{CambridgeUS} 
\usecolortheme[named=bleumarin]{structure} 
\useoutertheme[compress]{smoothbars}

% Theme finetuning
\setbeamertemplate{items}[ball]
\setbeamertemplate{blocks}[rounded][shadow=true]
\setbeamertemplate{navigation symbols}{}
\setbeamertemplate{headline}{}  


%%%%%%%%%%%%%%%%%%%%%%%%%%%%%%%%%%%%%%%%%%%%%%%%%%%%%%%%%%%%%%%%%%%%%% 
% TITLE PAGE
\title[Demonstratoare în Racket]{Crafting Provers in Racket}
\author[Adrian Manea]{absolvent: Adrian Manea \\ coordonator: Traian Șerbănuță}
\institute{510, SLA}

\date{}

\begin{document}

\maketitle

% SLIDES START HERE
%%%%%%%%%%%%%%%%%%%%%%%%%%%%%%%%%%%%%%%%%%%%%%%%%%%%%%%%%%%%%%%%%%%%%% 
\fr{
  \ft{Introducere și motivație}

  \lin{1}{Scurtă introducere în \textbf{teoria tipurilor} (MLTT) și verificări %
    folosind \textbf{Racket}}{3}

  \lin{2}{\textbf{MLTT} pentru că folosește logica intuiționistă și stă la baza HoTT}{3}

  \lin{3}{\textbf{Racket} pentru că este pornit din Lisp (preferință personală) și creat
    expres pentru cercetări în limbaje de programare}{3}
}

\fr{
  \ft{Plan}

  \begin{enumerate}
    \ite{1}{Elemente de bază pentru MLTT: $ \lambda $-calcul și tipuri dependente}
    \ite{2}{\textsc{Proust}, \emph{``a nano proof assistant''}}
    \ite{3}{\textsc{Pie} pentru tipuri dependente}
  \end{enumerate}

  % \vspace{1cm}
  
  % \lin{4}{Referințe principale:}{0}
  % \begin{enumerate}{\small
  %   \ite{5}{MLTT: \cite{mltt}, \cite{hott}, \cite{pmltt};}
  %   \ite{6}{Proust: \cite{proust};}
  %   \ite{7}{Pie: \cite{typer};}
  %   \ite{8}{Racket: \cite{racket}, \cite{htdp}}}
  % \end{enumerate}
}

\fr{
  \ft{Teoria tipurilor în formularea P.\ Martin-L\"{o}f ($\sim$1980)}

  \lin{1}{Dezvoltarea istorică (\cite{collins}):}{0}
  \begin{itemize}
    \ite{2}{paradoxul lui Russell;}
    \ite{3}{probleme de limbaj: Frege, Wittgenstein, Carnap;}
    \ite{4}{$\dots$ corespondența Curry-Howard(-Lambek).}
  \end{itemize}
  
  \lin{5}{Distincții clare față de teoria mulțimilor $ \Rightarrow $ intenții %
    fundamentale, accentuate de \emph{teoria toposurilor} (categorii + logică)}{3}

  \lin{6}{\emph{Abordare intuiționistă:}}{0}
  \begin{itemize}
    \ite{7}{existențialii trebuie instanțiați finitist;}
    \ite{8}{terțul exclus dispare;}
    \ite{9}{demonstrații bazate pe judecăți și reguli de inferență.}
  \end{itemize}
}

\fr{
  \ft{Teoria tipurilor în formularea P.\ Martin-L\"{o}f ($\sim$1980)}

  \lin{1}{Formulare (sintaxă) abstractă, interpretări (semantici) diverse:}{0}
  \begin{itemize}
    \ite{2}{propoziții ca tipuri (Curry-Howard, Wadler et al.)}
    \ite{3}{categorii cartezian închise (Lambek)}
    \ite{4}{probleme și soluții (Brouwer-Heyting-Kolmogorov)}
  \end{itemize}

  \lin{5}{
    \begin{figure}[!htb]
      \centering
      \begin{tabular}{l | l | l}
        A mulțime & $ a \in A $ & implică \\
        \hline
        $ A $ mulțime & $ a $ element al $ A $ & $ A $ nevidă \\
        $ A $ propoziție & $ a $ demonstrație constructivă a $ A $ & $ A $ este adevărată \\
        $ A $ intenție & $ a $ metodă de a îndeplini $ A $ & $ A $ realizabilă \\
        $ A $ problemă & $ a $ metodă de rezolvare a $ A $ & $ A $ are soluție
      \end{tabular}
      \caption{Interpretări diverse ale $ a \in A $, \cite[p.\ 4]{mltt}}
    \end{figure}
  }{2}
}

\fr{
  \ft{Teoria tipurilor în formularea P.\ Martin-L\"{o}f ($\sim$1980)}

  \lin{1}{\textbf{Tipuri dependente:}}{0}
  \lin{2}{Apărute pentru interpretarea cuantificatorilor.}{0}

  \lin{3}{\emph{Expresii parametrizate}, i.e.\ %
    expresii ,,la fel'' sintactic, cu interpretări diferite.}{0}

  \vspace{0.5cm}
  
  \lin{4}{Judecăți primitive (stil Martin-L\"{o}f):}{0}
  \begin{itemize}
    \ite{5}{$ A $ este un \emph{tip corect format} în contextul $ \Gamma $: $ \Gamma \vdash A \text{ type} $;}
    \ite{6}{$ A $ și $ B $ sînt \emph{tipuri identice} în contextul $ \Gamma $: $ \Gamma \vdash A \equiv B \text{ type} $;}
    \ite{7}{$ A $ este un \emph{termen corect format}, de tip $ A $, în contextul $ \Gamma $:
      $ \Gamma \vdash a : A $;}
    \ite{8}{$ a $ și $ b $ sînt \emph{termeni egali} de tip $ A $ în contextul $ \Gamma $:
      $ \Gamma \vdash a \equiv b : A $.}
  \end{itemize}

  \vspace{0.5cm}

  \lin{9}{Context = asumpții de tipizare, de forma: \[ \Gamma = x_1 : A_1, x_2 : A_2(x_1), \dots, %
      x_n : A_n(x_1, \dots, x_{n-1}) \]}{0}
}

\fr{
  \ft{Lambda calcul fără tipuri ($\sim$1930)}

  \lin{1}{Intuitiv: un formalism pentru ,,funcții anonime''.}{1}

  \lin{2}{Gramatica BNF: \[ t ::= x \mid \lambda x. t \mid tt \]}{0}

  \begin{itemize}
    \ite{3}{variabile libere $ x $;}
    \ite{4}{lambda-abstracții (leagă variabila $ x $ în termenul $ t $);}
    \ite{5}{aplicări ale unui termen pe alt termen.}
  \end{itemize}

  \vspace{1cm}

  \lin{6}{Exemple: $ x $, $ \lambda x . 5 $, $ (\lambda x . 5) z $, %
    $ (\lambda x . 5)(\lambda z . (z + 1)) $ etc.}{1}

  \lin{7}{Astăzi, majoritatea limbajelor de programare au o formă de a declara
    funcții anonime cu expresii \texttt{lambda}.}{0}
}

\fr{
  \ft{Tipuri funcționale}

  \lin{1}{$ A, B \text{ type} $; asociem elementele $ \Rightarrow $ \emph{tipul funcțional} $ A \to B $.}{1}

  \lin{2}{Inferențele pentru tipuri funcționale se fac cu \emph{lambda calcul}.}{0}

  \lin{3}{\[ \begin{prooftree}
        \hypo{\Gamma \vdash B \text{ type}}
        \hypo{\Gamma, x : A \vdash b(x) : B}
        \infer2{\Gamma \vdash \lambda x . b(x) : A \to B}
      \end{prooftree} \quad (\lambda) \]}{0}

  \lin{4}{\[ \begin{prooftree}
        \hypo{\Gamma \vdash f : A \to B}
        \infer1{\Gamma, x : A \vdash f(x) : B}
      \end{prooftree}\quad \text{(eval)} \]}{0}

  \lin{5}{\[\begin{prooftree}
        \hypo{\Gamma \vdash B \text{ type}}
        \hypo{\Gamma, x : A \vdash b(x) : B}
        \infer2{\Gamma, x : A \vdash (\lambda y . b(y))(x) \equiv b(x) : B}
      \end{prooftree} \quad (\beta) \]}{0}
}

\fr{
  \ft{Tipuri dependente: $ \Pi $ (produs)}

  \lin{1}{Ideea: \emph{familii de tipuri}, parametrizate de un tip anume.}{1}

  \lin{2}{Pentru fiecare $ x : A $, asociem \emph{cîte un tip} $ B(x) $.}{1}

  \lin{3}{Notația $ \prod_{(x : A)} B(x) $, numite și \emph{funcții dependente}.}{1}

  \lin{4}{Interpretare: cuantificarea universală, %
    $ (\forall x : A)B(x) \leadsto \prod_{(x : A)} B(x) $.}{1}

  \vspace{0.5cm}

  \lin{5}{Exemplu: \textbf{numerele naturale}.}{1}

  \lin{6}{\emph{Principiul de inducție}, ca regulă de inferență:}{0}

  \lin{7}{\small\[\begin{prooftree}
        \hypo{\Gamma, n : \mathbb{N} \vdash P \text{ type}}
        \hypo{\Gamma \vdash p_0 : P(0)}
        \hypo{\Gamma \vdash p_S : \prod_{(n : \mathbb{N})} \left(P(n) \to P(S(n))\right)}
        \infer3{\Gamma \vdash \dr{ind}_{\mathbb{N}} (p_0, p_S) : \prod_{(n : \mathbb{N})} P(n)}.
      \end{prooftree}(\mathbb{N}-\text{ind}).\]}{1}
}

\fr{
  \ft{Tipuri dependente: $ \Sigma $ (sumă)}

  \lin{1}{Ideea: Pentru $ a : A $, asociem cîte un tip $ B(a) \Rightarrow \sum_{(a : A)} B(a) $}{2}

  \lin{2}{Interpretare: cuantificarea existențială, $ (\exists x : A) B(a) \leadsto \sum_{(x : A)} B(a)$.}{0}

  \vspace{1cm}

  \lin{3}{\textbf{Exemplu:} Fie $ x : \mathbb{N} $ fixat. Definim predicatul $ P(y): d \cdot y = x $.}{1}

  \lin{4}{Formăm tipul sumă $ \sum_{(y : \mathbb{N})} P(y) $, populat de:}{1}

  \begin{itemize}
    \ite{5}{elemente de tip $ y : \mathbb{N} $;}
    \ite{6}{pentru fiecare $ y $, demonstrații ale $ P(y) $, i.e.\ cîte un $ d $, %
      \emph{dacă există}.}
  \end{itemize}
}

\fr{
  \ft{Proust (2016)}

  \lin{1}{Demonstrator DIY, cu accent didactic.}{2}

  \lin{2}{Verifică tipizarea (în context) pentru:}{0}
  \begin{itemize}
    \ite{3}{tipizări explicite (\emph{type annotations}), \texttt{Ann (expr type)};}
    \ite{4}{tipuri funcționale, \texttt{Arrow (domain codomain)};}
    \ite{5}{aplicări de funcții, \texttt{App (func arg)};}
    \ite{6}{expresii lambda, \texttt{Lam (var body)};}
  \end{itemize}

  \vspace{0.5cm}

  \lin{7}{Exemplu: \[ \small\texttt{(check-expect (check-proof '((lambda x => x) : (A -> A))) true)} \]}{0}
}

\fr{
  \ft{Pie (2018)}

  \lin{1}{Limbaj specific (DSL), cu accent pe \emph{tipizare statică} și \emph{tipuri dependente}.}{2}

  \lin{2}{Implementat inițial în Racket, mutat apoi în Haskell (\texttt{pie-hs}).}{2}

  \lin{3}{Exemple:}{0}
  \begin{itemize}
    \ite{4}{Rezervarea tipului: {\small\texttt{(claim one Nat)}};}
    \ite{5}{Definiția: {\small\texttt{(define one (add1 zero))}};}
    \ite{6}{Tipizarea explicită: {\small\texttt{(the Nat (add1 one))}};}
    \ite{7}{Tipuri $ \Pi $: {\small\texttt{(Pi ((A U) (D U)) (-> (Pair A D) (Pair D A)))}}.}
  \end{itemize}
}

%%%%%%%%%%%%%%%%%%%%%%%%%%%%%%%%%%%%%%%%%%%%%%%%%%%%%%%%%%%%%%%%%%%%%% 

% Bibliography
\begin{frame}[allowframebreaks]

  \ft{Bibliografie și lecturi suplimentare}
  
  \bibliographystyle{apalike}
  \bibliography{slides.bib}

  \nocite{*}
\end{frame}

\end{document}

