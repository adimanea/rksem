% ! TEX root = ../redex.tex
\chapter*{Introduction and Motivation}

The starting point for the present work was my motivation of learning
the programming language Racket. Having a background in mathematics,
the lambda calculus and the function programming languages appealed to
me instantly and once I got acquainted with Emacs and the Lisps, my
interest grew. Furthermore, the excellent theoretical foundations laid
by J.\ McCarthy in \cite{mccarthy62} and \cite{mccarthy61}, described
by some computer scientists as having introduced a paradigm shift in
theoretical computer science that's comparable to the non-Euclidean
geometry revolution proved to be an excellent introduction and motivation
for wanting to learn more about the Lisp family of programming language.
Before long, I discovered the influential \cite{sicp} and saw how an
apparently simple programming language such as Scheme can reveal wonderful
constructions of various degrees of abstraction.

Doing some more reading in the world of Lisp dialects, I have discovered
Racket, whose appeal was instantaneous to me, since it is so often described
not as a general purpose programming language derived from Scheme, but rather
as a toolbox for constructing languages to solve various problems. In fact,
as I saw it, it offers the necessary items for one to properly understand,
craft and teach languages that exhibit particular behaviour.

This is precisely the aim of the current work. Given the inherent shortcomings
of Scheme, assumed by its creators for the sake of simplicity and ease of
use and extension, and not trying to delve into the whole \qq{zoo} of Lisp
dialects, I found Racket to be the most appropriate object of study for my
purposes. These are to study semantic aspects of (mostly) functional programming
languages, as well as type theory and related formal methods, which Racket
has the flexibility to allow, all while using a mild variation of the syntax
that was established ever since McCarthy's (Common) Lisp.

However, this work is not a Racket manual and given its sheer flexibility
and array of features, it is beyond my scope to thoroughly explore this
language-toolbox, at least not when confined within the scope of this
dissertation. Therefore, the precise topic I chose to focus on this work is
that of \textbf{PLT Redex} (henceforth called \textbf{Redex} in short, since
PLT is the name of the research group that started it, continuing efforts from
PLT Scheme, which was to become Racket).

Inspired by the great article and presentation of B.\ Findler at POPL 2012
(\cite{popl}), I will be presenting the main features of Redex (in Racket,
as it is implemented), provide some examples and try to show how it can be
used not only to create toy languages, but also some which come with included
formal methods of checking correctness (a rather vague term which will be
detailed in due time).

\vspace{0.3cm}

The plan of the work is as follows. A short historical presentation and
preliminaries make up the first part of the dissertation. Then a Scheme
and Racket \emph{crash course} will follow, focusing on specific aspects
that will come in handy when discussing Redex, such as \texttt{call/cc} and
\texttt{amb}. The main part of the work will then contain the actual
presentation of Redex, along with some standard examples that the authors
provided (\cite{amb,long,sewpr}). Finally, further examples are provided,
which exhibit features of interest, related mostly to type theory.

The work is in fact part of a more elaborate plan, which will be detailed
in the final section of this dissertation.

\vspace{0.3cm}

It is also worth mentioning that the PLT group (\cite{pltgrp}) used Redex
as a starting point for their work on a language-creating toolbox and
their current focus is on Pyret, described as \emph{a programming language %
  designed to serve as an outstanding choice for programming education while %
  exploring the confluence of scripting and functional programming}
(\cite{pyret}). However, since as we will detail, the wider focus is on
Racket, we will not touch on this subject here.

\todo[inline,noline,backgroundcolor=green!40]{change the bib style %
(especially to go well with websites)}



%%% Local Variables:
%%% mode: latex
%%% TeX-master: "../redex"
%%% End:
